\documentclass{beamer}

\usepackage{../../latex_style/beamerthemeExecushares}
\usepackage{../../latex_style/notations}
\usepackage{dsfont}

\title{Recitation 12}
\author{Carles Domingo}
\date{Fall 2020}


\begin{document}

\frame{\titlepage} 

\setcounter{showProgressBar}{0}
\setcounter{showSlideNumbers}{1}

\begin{frame}[t]
\frametitle{Exercise 4, 2018 review}
Let $A \in \R^{n \times n}$ be symmetric matrix with eigenvalues $\lambda_1, \dots, \lambda_n$. Prove that $\|Ax\| \leq \max_i |\lambda_i| \|x\|$ for any $x \in \R^n$.
\pause
\end{frame}

\begin{frame}[t]
\frametitle{Exercise 8, 2018 review}
Suppose $A \in \R^{m \times n}$ has rank $m$. Prove that $A A^{\top}$ is invertible.
\pause
\end{frame}

\begin{frame}[t]
\frametitle{Exercise 9, 2018 review}
\vspace{-5pt}
Consider the optimization problem
\begin{align*}
\text{minimize }_x &\|x\|_2 \\
\text{subject to } &Ax = b 
\end{align*}
where $A \in \R^{m \times n}$ and $b \in \R^m$ are fixed and $b \in \text{Im}(A)$.
\begin{enumerate}[(a)]
\item Prove that any minimizer $x^{*}$ must belong to $\text{Im}(A^{\top})$.
\item Give a formula for the minimizer $x^{*}$, and show it is unique.
\end{enumerate}
\pause
\pause
\end{frame}

\begin{frame}[t]
\frametitle{Exercise 10, 2018 review}
\vspace{-5pt}
Let $A \in \R^{n \times m}$ and $B \in \R^{n \times k}$ and define the block matrix $C \in \R^{n \times (m+k)}$ by 
\begin{align*}
C = \begin{bmatrix}
A & B
\end{bmatrix}.
\end{align*}
Either prove the following statement or give a counterexample: 
\begin{align*}
\text{Rank}(C) = \text{Rank}(A) + \text{Rank}(B).
\end{align*}
\pause
\end{frame}

\begin{frame}[t]
\frametitle{Exercise 20, 2018 review}
\vspace{-5pt}
Let $A \in \R^{n \times n}$ be a square matrix with the (unusual) property that the image space (or column space) $\text{Im}(A)$ of $A$ is equal to its kernel (or nullspace) $\text{Ker}(A)$.
\begin{enumerate}[(a)]
\item What can you say about $A^2$? 
\item Give an example of such an $A$.
\end{enumerate}
\pause
\end{frame}

\begin{frame}[t]
\frametitle{Exercise 25, 2018 review}
\vspace{-5pt}
Let $A = U \Sigma V^{\top}$ denote the Singular Value Decomposition of $A \in \R^{m \times n}$. Let $A' = U \Sigma' V^{\top}$ where $\Sigma'$ is obtained from $\Sigma$ by replacing every entry by zero except for the entry corresponding to the largest singular value.
\begin{enumerate}[(a)]
\item Show that $A'$ is the best rank 1 approximation of $A$ in the Frobenius norm, meaning that $A'$ is the solution to $\min_{B: \text{rank}(B)=1} \|B - A\|_F$.
\item Show that $A'$ is the best rank 1 approximation of $A$ in the spectral norm, meaning that $A'$ is the solution to $\min_{B: \text{rank}(B)=1} \|B - A\|$.
\end{enumerate}
\end{frame}

\begin{frame}[t]
\frametitle{Exercise 25, 2018 review}
\vspace{-5pt}
(a) Show that $A'$ is the best rank 1 approximation of $A$ in the Frobenius norm, meaning that $A'$ is the solution to $\min_{B: \text{rank}(B)=1} \|B - A\|_F$.
\pause
\end{frame}

\begin{frame}[t]
\frametitle{Exercise 25, 2018 review}
\vspace{-5pt}
(b) Show that $A'$ is the best rank 1 approximation of $A$ in the spectral norm, meaning that $A'$ is the solution to $\min_{B: \text{rank}(B)=1} \|B - A\|$.
\pause
\end{frame}

\begin{frame}[t]
\frametitle{Exercise 0.9, 2019 review}
\vspace{-5pt}
For each of the following statement, say if they are true or false and justify your answer.
\begin{itemize}
\item If a continuous function $f : \R \rightarrow \R$ has a unique minimizer then f is convex.
\item If a continuous function $f : \R \rightarrow \R$ is such that there exists $x_0$ such that $f$ is decreasing on
$(-\infty, x_0]$ and increasing on $[x_0, +\infty)$ then $f$ is convex.
\item A twice differentiable function $f : \R \rightarrow \R$ whose derivative $f'$ is non-decreasing is convex.
\end{itemize}
\end{frame}

\begin{frame}[t]
\frametitle{Exercise 0.9, 2019 review}
\vspace{-5pt}
\begin{itemize}
\item If a continuous function $f : \R \rightarrow \R$ has a unique minimizer then f is convex.
\end{itemize}
\end{frame}

\begin{frame}[t]
\frametitle{Exercise 0.9, 2019 review}
\vspace{-5pt}
\begin{itemize}
\item If a continuous function $f : \R \rightarrow \R$ is such that there exists $x_0$ such that $f$ is decreasing on
$(-\infty, x_0]$ and increasing on $[x_0, +\infty)$ then $f$ is convex.
\end{itemize}
\end{frame}

\begin{frame}[t]
\frametitle{Exercise 0.9, 2019 review}
\vspace{-5pt}
\begin{itemize}
\item A twice differentiable function $f : \R \rightarrow \R$ whose derivative $f'$ is non-decreasing is convex.
\end{itemize}
\end{frame}

\begin{frame}[t]
\frametitle{Exercise 0.10, 2019 review}
\vspace{-5pt}
Let $f : \R^n \rightarrow \R$ be a convex, differentiable function. Assume that there exist $x,y \in \R^n$ such that $\nabla f(x) = \nabla f(y) = 0$. Show that $\nabla f((x+y)/2) = 0$.
\end{frame}

\begin{frame}[t]
\pause
\pause
\pause
\pause
\end{frame}


\end{document}


