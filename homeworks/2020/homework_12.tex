\documentclass[11pt,nocut]{article}

\usepackage{../../latex_style/packages}
\usepackage{../../latex_style/notations}

\title{\vspace{-2.0cm}%
	Optimization and Computational Linear Algebra for Data Science\\
Homework 12: Gradient descent}
\date{}
\author{Due on December 13, 2020}
\setcounter{section}{12}

\begin{document}
\maketitle
\input{./preamble_homeworks.tex}

\vspace{-3mm}

\begin{problem}[2 points] The following plot shows the contour lines of a function $f:\R^2 \to \R$.
	\begin{figure}[H]
		\begin{center}
		\includegraphics[width=11cm]{contours.pdf}
		\end{center}
	\end{figure}
	\begin{enumerate}[label=\normalfont(\textbf{\alph*})]
		\item Give (approximately) the coordinates of the global/local minimizers/maximizers, saddle points of $f$.
		\item Assume that we run gradient descent to minimize $f$. Will gradient descent converge to the global minimizer of $f$ when initialized at point $\bbf{A}$ ? at point $\bbf{B}$ ?
	\end{enumerate}
\end{problem}

\vspace{5mm}

\begin{problem}[5 points]\label{p:grad}
	Let $M \in \R^{d \times d}$ be a positive semidefinite matrix, $b \in \R^d$ and $c \in \R$. We aim at minimizing the quadratic function
	$$
	f(x) = \frac{1}{2} x^{\sT} M x - \langle x,b \rangle + c
	$$
	using gradient descent. 
	We assume that $M$ is positive definite (i.e.\ all its eigenvalues are positive).
	We let $\lambda_1 \geq \lambda_2 \geq \cdots \geq \lambda_d >0$ be its eigenvalues and let $v_1, \dots, v_d$ be an orthonormal basis of $\R^d$ consisting of associated eigenvectors ($Mv_i = \lambda_i v_i$ for all $i$).
	We write $L = \lambda_1$ and $\mu = \lambda_d$.
	\\

	We consider standard gradient descent with constant step-size $\beta$:
$$
x_{t+1} = x_t - \beta \nabla f(x_t).
$$
	\begin{enumerate}[label=\normalfont(\textbf{\alph*})]
		\item Show that $f$ is $L$-smooth, $\mu$-strongly convex and that $x^* = M^{-1} b$ is the unique minimizer of $f$.
		\item We now study the convergence of gradient descent to $x^*$. Show that for all $t \geq 0$,
			$$
			x_{t+1} - x^* = \big(\Id - \beta M \big)(x_t - x^*).
			$$
		\item From now, we set $\beta = 1/L$. Deduce from the previous question that for all $t \geq 0$
			$$
			\|x_t - x^* \| \leq \Big(1- \frac{\mu}{L}\Big)^{\! t} \, \|x_0 - x^*\|.
			$$
		\item We would like now to have something more precise than the error bound of the previous question. We define $w_t \defeq x_t -x^*$. Let 
			$$
			\alpha_1(t) = \langle v_1, w_t \rangle, \cdots, \alpha_d(t) = \langle v_d, w_t \rangle
			$$
			be the coordinates of $w_t$ in the orthonormal basis $(v_1, \dots, v_d)$.
			For $i \in \{1, \dots, d\}$, express $\alpha_i(t)$ in terms of $t,\lambda_i,L$ and $\alpha_i(0)$. 
		\item Using the previous question, justify the following sentence:
			\begin{center}
			\emph{
				<< Gradient descent converges towards the minimizer faster in directions given  by the eigenvectors of the Hessian of $f$ corresponding to large eigenvalues than in directions corresponding to eigenvectors with small eigenvalues.>>
			}
			\end{center}
		\item Show that for all $t \geq 0$
			$$
			\|x_t - x^* \| = \sqrt{\sum_{i=1}^d \Big(1-\frac{\lambda_i}{L}\Big)^{\!2t} \big\langle v_i, x_0-x^* \big\rangle^2}.
			$$
	\end{enumerate}
\end{problem}

\vspace{5mm}

\begin{problem}[3 points]
	In this problem, you will implement and compare gradient descent with or without momentum to minimize the Ridge cost function:
	$$
	f(x) = \frac{1}{2} \|Ax-y\|^2 + \frac{\lambda}{2} \|x\|^2.
	$$
	All the instructions and questions are in the Jupyter notebook \texttt{gradient\_descent.ipynb}.
	\\

	\textbf{It is intended that you code in Python and use the provided Jupyter Notebook. Please only submit a pdf version of your notebook (right-click $\to$ `print' $\to$ `Save as pdf').}
\end{problem}


\vspace{5mm}

\begin{problem}[$\star$]
	We take exactly the same setting of Problem~\ref{p:grad}, but we now consider gradient descent with momentum:
$$
x_{t+1} = x_t - \beta \nabla f(x_t) + \gamma(x_t -x_{t-1}),
$$
for $t \geq 1$,
where we take
$$
\beta = \frac{4}{(\sqrt{L} + \sqrt{\mu})^2}
\qquad \text{and} \qquad
\gamma = \left(\frac{\sqrt{L}-\sqrt{\mu}}{\sqrt{L}+\sqrt{\mu}}\right)^{\! 2}.
$$
Show that the $\alpha_i(t) \defeq \langle v_i, x_t - x^* \rangle$ satisfy a second order linear recurrence relation (as a sequence indexed by $t$). Using this relation, show that for all $t \geq 0$
$$
|\alpha_i(t)| \leq C_i \left(\frac{\sqrt{L}-\sqrt{\mu}}{\sqrt{L} + \sqrt{\mu}}\right)^{\! t}
$$
where $C_i$ is a constant that does not depend on $t$, but that may depend on $x_0,x_1, \mu$ and $L$ (a precise expression of $C_i$ is not expected). Deduce that for all $t \geq 0$
$$
\|x_t - x^*\| \leq C \left(\frac{\sqrt{L}-\sqrt{\mu}}{\sqrt{L} + \sqrt{\mu}}\right)^{\! t}
$$
where $C$ is a constant that does not depend on $t$.
\end{problem}

%\begin{problem}[$\star$]
	%We take exactly the same setting of Problem~\ref{p:grad}, but we now consider a variant of gradient descent:
%$$
%x_{t+1} = x_t - \beta \big(\nabla f(x_t) + z_t\big)
%$$
%where the $z_1,z_2, \dots, z_t, \dots$ are i.i.d.\ standard Gaussian vectors (of dimension $d$) that account for some noise.
%This is a simplistic model to study stochastic gradient descent: indeed the gradient used by SGD on minibatches can be interpreted as noisy version of the full gradient $\nabla f$. This is why we model the behavior of SGD by adding noise to $\nabla f$.
%\\

%\end{problem}


\vspace{1cm}
\centerline{\pgfornament[width=7cm]{87}}

%\bibliographystyle{plain}
%\bibliography{./references.bib}
\end{document}
